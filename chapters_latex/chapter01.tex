\section*{Chapter 1}

\subsection*{Exercise 1.1: Self-Play} 
Suppose, instead of playing against a random opponent, the
reinforcement learning algorithm described above played against itself, with both sides
learning. What do you think would happen in this case? Would it learn a different policy
for selecting moves? 

\subsubsection*{Solution:}
\begin{itemize}
    \item Early games would be random and unstructured.
    \item Over time, the agent would improve by reinforcing winning moves and avoiding losing ones.
    \item Eventually, the agent would converge toward optimal play, leading to most games ending in a draw.
    \item The RL agent would effectively learn to play tic-tac-toe perfectly, unable to lose but also unable to win against a similarly skilled opponent (itself).
  \end{itemize}

\subsection*{Exercise 1.2 Symmetries} 
Many tic-tac-toe positions appear different but are really
the same because of symmetries. How might we amend the learning process described
above to take advantage of this? In what ways would this change improve the learning
process? Now think again. Suppose the opponent did not take advantage of symmetries.
In that case, should we? Is it true, then, that symmetrically equivalent positions should
necessarily have the same value?

\subsubsection*{Solution:}
\begin{itemize}
    \item Rather than treating each possible board configuration as unique, the RL agent should group board positions that are equivalent under symmetry. This would reduce the number of unique states the agent needs to learn/investigate.
    \item No, we shouldn't exploit symmetries if the opponent doesn't exploit them either. If the opponent has a policy that behaves differently at states that are symmetric to each other, then the agent couldn't expoit this if it treated symmetric states the same. In this case symmetrically equivalent positions should not have the same value.
  \end{itemize}

\subsection*{Exercise 1.3 Greedy Play}
Suppose the reinforcement learning player was greedy, that is,
it always played the move that brought it to the position that it rated the best. Might it learn to play better, or worse, than a nongreedy player? What problems might occur? 

\subsubsection*{Solution:}
\begin{itemize}
    \item Case 1, The opponent is deterministic (always plays the same move at a certain state):\\
    In the beginnig, all of the non-winning states have a 50\% rating, so the greedy player would choose a random move. If this move leads to winning, its estimate increases, so in the following rounds the agent will choose it again and again. If it leads to losing, then its estimate decreases, so the agent will choose another move in the following rounds. This way the agent will find a set of steps that will always lead to winning (if they exist).
    \item Case 2, The opponent is non-deterministic:\\
    Suppose at first try the agent finds a set of steps which always produce an estimate that is $>50\%$. The greedy agent will always choose these steps (the others were initialized at 50\%), but there might be a set of steps that achieve better winrate. In this case there is no guarantee that the agent finds the best policy.\\
    (This is a special case, the winrate can be arbitrarily small)  
  \end{itemize}

\subsection*{Exercise 1.4 Learning from Exploration}
Suppose learning updates occurred after all
moves, including exploratory moves. If the step-size parameter is appropriately reduced
over time (but not the tendency to explore), then the state values would converge to
a different set of probabilities. What (conceptually) are the two sets of probabilities
computed when we do, and when we do not, learn from exploratory moves? Assuming
that we do continue to make exploratory moves, which set of probabilities might be better
to learn? Which would result in more wins?

\subsubsection*{Solution:}
\begin{itemize}
    \item If we don't update after exploratory moves: the value is the probability that a greedy agent wins from that state.
    \item If we update after exploratory moves: the value is the probability that an agent that is prone to explore wins from that state.
    \item If we continue to explore, then we should include the exploratory moves into the updates.
\end{itemize}

\subsection*{Exercise 1.5 Other Improvements} 
Can you think of other ways to improve the reinforcement learning player? Can you think of any better way to solve the tic-tac-toe problem
as posed?

\subsubsection*{Solution:}
\begin{itemize}
    \item Incorporate domain knowledge (Heuristics and Rules): e.g. always take center.
    \item Use of Value Function Approximation: use a neural network or other function approximator to estimate the value function.
\end{itemize}